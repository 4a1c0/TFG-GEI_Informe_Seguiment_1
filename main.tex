\documentclass[conference,onecolumn, catalan]{IEEEtran}
\IEEEoverridecommandlockouts
% The preceding line is only needed to identify funding in the first footnote. If that is unneeded, please comment it out.
\usepackage{comment}
\usepackage{hyperref}
\usepackage{cite}
\usepackage{amsmath,amssymb,amsfonts}
\usepackage{algorithmic}
\usepackage{textcomp}
\usepackage{xcolor}
\def\BibTeX{{\rm B\kern-.05em{\sc i\kern-.025em b}\kern-.08em
    T\kern-.1667em\lower.7ex\hbox{E}\kern-.125emX}}
\newtheorem{theorem}{Teorema}
\usepackage{listings}
\usepackage{babel}
\usepackage{geometry}% http://ctan.org/pkg/geometry
\usepackage{graphicx}% http://ctan.org/pkg/graphicx
\usepackage{color} %red, green, blue, yellow, cyan, magenta, black, white
\definecolor{mygreen}{RGB}{28,172,0} % color values Red, Green, Blue
\definecolor{mylilas}{RGB}{170,55,241}
\lstset{language=Matlab,%
    %basicstyle=\color{red},
    breaklines=true,%
    morekeywords={matlab2tikz},
    keywordstyle=\color{blue},%
    morekeywords=[2]{1}, keywordstyle=[2]{\color{black}},
    identifierstyle=\color{black},%
    stringstyle=\color{mylilas},
    commentstyle=\color{mygreen},%
    showstringspaces=false,%without this there will be a symbol in the places where there is a space
    numbers=left,%
    basicstyle=\small,
    numbers = none,
    %numberstyle=none,% size of the numbers
    %numbersep=2pt, % this defines how far the numbers are from the text
    emph=[1]{for,end,break},emphstyle=[1]\color{red}, %some words to emphasise
    %emph=[2]{word1,word2}, emphstyle=[2]{style},    
}

\newenvironment{changemargin}[2]{%
\begin{list}{}{%
\setlength{\topsep}{0pt}%
\setlength{\leftmargin}{#1}%
\setlength{\rightmargin}{#2}%
\setlength{\listparindent}{\parindent}%
\setlength{\itemindent}{\parindent}%
\setlength{\parsep}{\parskip}%
}%
\item[]}{\end{list}}

\title{Primer Informe de Progrés del TFG: \\ \vspace{0.2cm} {\huge \textit{ Desenvolupament d'un ``core" didàctic de RISC-V\ }} }
\author{
\IEEEauthorblockN{Pau Casacuberta Orta}
\IEEEauthorblockA{
\textit{Autonomous University of Barcelona}\\
Cerdanyola del Vallès, Barcelona 08193\\
pau.casacubertao@e-campus.uab.cat\\}}

\usepackage[owncaptions, tablegrid]{vhistory}
\renewcommand{\vhhistoryname}{Històric de versions}
\renewcommand{\vhversionname}{Versió}
\renewcommand{\vhdatename}{Data}
\renewcommand{\vhauthorname}{Autor(s)}
\renewcommand{\vhchangename}{Descripció}

\begin{document}

\maketitle

\begin{versionhistory}
    \vhEntry{0.1}{07.17.19}{Pau}{Creació del document}
    %\vhEntry{1.1}{06.10.19}{Pau}{Modificacions proposades pel tutor (Raimon)}
\end{versionhistory}

\begin{abstract}
La finalitat d'aquest document és el de consignar els avenços efectuats en el desenvolupament del treball.
\end{abstract}

\section{Seguiment}

A data de 7 de novembre de 2019 s'ha consolidat part de les tasques per al desenvolupament del projecte. 
Les tasques de recerca inicial, i la majoria de les de programació s'han dut a terme. Com a resultat d'això es pot veure en el GitHub\cite{casacuberta_orta_4a1c0/rv32i-verilog_2019} com s'han implementat els blocs necessaris per a que el core pugui executar tot el repertori RV32I (a excepció de les crides \textit{FENCE} ja que no es faran servir en aquesta implementació). 
Així com els tests corresponents per a verificar el correcte funcionament al executar instruccions.

Al desenvolupar les tasques anteriorment esmentades s'ha notat que fan falta ajustar el nombre d'hores de dedicació a diferents apartats 

\section{Canvis}

Veient el desenvolupament del projecte, i per reduir riscos per la seva finalització, s'ha decidit treure dels objectius la implementació del core amb pipeline degut a que aquesta pot comportar un temps elevat per a ésser completada en el calendari establert.


\begin{figure}[!ht]
\centering
%\includegraphics[width=\textwidth]{}
\caption{Gantt Diagram. This figure lists all tasks, milestones and summaries of the project}
\label{fig:gantt}
\end{figure}


\subsection{Metodologia}

En Relació a la metodologia per a desenvolupar el projecte es la 


\section{Conclusion}

The exercise of planing this project gave us experience at using the tool Microsoft Project to organize a generic project.



\bibliographystyle{IEEEtran}
\bibliography{references.bib}


\end{document}
